%!TEX TS-program = xelatex
\documentclass[notes,12pt, aspectratio=169]{beamer}

\usepackage{amsmath,amsfonts,amssymb,amsthm,mathtools}  % пакеты для математики
\usepackage{minted}

\usepackage[english, russian]{babel} % выбор языка для документа
\usepackage[utf8]{inputenc} % задание utf8 кодировки исходного tex файла
\usepackage[X2,T2A]{fontenc}        % кодировка

\usepackage{fontspec}         % пакет для подгрузки шрифтов
\setmainfont{Helvetica}  % задаёт основной шрифт документа

% why do we need \newfontfamily:
% http://tex.stackexchange.com/questions/91507/
\newfontfamily{\cyrillicfonttt}{Helvetica}
\newfontfamily{\cyrillicfont}{Helvetica}
\newfontfamily{\cyrillicfontsf}{Helvetica}

\usepackage{unicode-math}     % пакет для установки математического шрифта
% \setmathfont{Neo Euler} % шрифт для математики

\usepackage{polyglossia}      % Пакет, который позволяет подгружать русские буквы
\setdefaultlanguage{russian}  % Основной язык документа
\setotherlanguage{english}    % Второстепенный язык документа

% Шрифт для кода
\setmonofont[Scale=0.85]{Monaco}
\usepackage{verbments}

\usepackage{pgfpages}
% These slides also contain speaker notes. You can print just the slides,
% just the notes, or both, depending on the setting below. Comment out the want
% you want.
%\setbeameroption{hide notes} % Only slide
%\setbeameroption{show only notes} % Only notes
%\setbeameroption{show notes on second screen=right} % Both

\usepackage{array}

\usepackage{tikz}
\usepackage{verbatim}
\setbeamertemplate{note page}{\pagecolor{yellow!5}\insertnote}
\usetikzlibrary{positioning}
\usetikzlibrary{snakes}
\usetikzlibrary{calc}
\usetikzlibrary{arrows}
\usetikzlibrary{decorations.markings}
\usetikzlibrary{shapes.misc}
\usetikzlibrary{matrix,shapes,arrows,fit,tikzmark}

\usepackage{hyperref}
\usepackage{lipsum}
\usepackage{multimedia}
\usepackage{multirow}
\usepackage{dcolumn}
\usepackage{bbm}
\newcolumntype{d}[0]{D{.}{.}{5}}

\usepackage{changepage}
\usepackage{appendixnumberbeamer}
\newcommand{\beginbackup}{
   \newcounter{framenumbervorappendix}
   \setcounter{framenumbervorappendix}{\value{framenumber}}
   \setbeamertemplate{footline}
   {
     \leavevmode%
     \hline
     box{%
       \begin{beamercolorbox}[wd=\paperwidth,ht=2.25ex,dp=1ex,right]{footlinecolor}%
%         \insertframenumber  \hspace*{2ex} 
       \end{beamercolorbox}}%
     \vskip0pt%
   }
 }
\newcommand{\backupend}{
   \addtocounter{framenumbervorappendix}{-\value{framenumber}}
   \addtocounter{framenumber}{\value{framenumbervorappendix}} 
}

% для имитации питоновского синтаксиса 
\newcommand{\pgr}[1]{{\color{green} \textbf{#1}}}


%%%%%%%%%% Работа с картинками %%%%%%%%%
\usepackage{graphicx}                  % Для вставки рисунков
\usepackage{graphics}
\graphicspath{{images/},{imagess/}}    % можно указать папки с картинками
\usepackage{wrapfig}                   % Обтекание рисунков и таблиц текстом

\usepackage[space]{grffile}
\usepackage{booktabs}

% These are my colors -- there are many like them, but these ones are mine.
\definecolor{blue}{RGB}{0,114,178}
\definecolor{red}{RGB}{213,94,0}
\definecolor{yellow}{RGB}{240,228,66}
\definecolor{green}{RGB}{0,128, 0}

\hypersetup{
  colorlinks=false,
  linkbordercolor = {white},
  linkcolor = {blue}
}


%% I use a beige off white for my background
\definecolor{MyBackground}{RGB}{255,253,218}

%% Uncomment this if you want to change the background color to something else
%\setbeamercolor{background canvas}{bg=MyBackground}

%% Change the bg color to adjust your transition slide background color!
\newenvironment{transitionframe}{
  \setbeamercolor{background canvas}{bg=yellow}
  \begin{frame}}{
    \end{frame}
}

\setbeamercolor{frametitle}{fg=blue}
\setbeamercolor{title}{fg=black}
\setbeamertemplate{footline}[frame number]
\setbeamertemplate{navigation symbols}{} 
\setbeamertemplate{itemize items}{-}
\setbeamercolor{itemize item}{fg=blue}
\setbeamercolor{itemize subitem}{fg=blue}
\setbeamercolor{enumerate item}{fg=blue}
\setbeamercolor{enumerate subitem}{fg=blue}
\setbeamercolor{button}{bg=MyBackground,fg=blue,}


% If you like road maps, rather than having clutter at the top, have a roadmap show up at the end of each section 
% (and after your introduction)
% Uncomment this is if you want the roadmap!
% \AtBeginSection[]
% {
%    \begin{frame}
%        \frametitle{Roadmap of Talk}
%        \tableofcontents[currentsection]
%    \end{frame}
% }
\setbeamercolor{section in toc}{fg=blue}
\setbeamercolor{subsection in toc}{fg=red}
\setbeamersize{text margin left=1em,text margin right=1em} 

% списки, которые растягиваются на всю величину слайда 
\newenvironment{wideitemize}{\itemize\addtolength{\itemsep}{10pt}}{\enditemize}



\title[]{\textcolor{blue}{Глубокое обучение и вообще}}
\author{Ульянкин Филипп}
\date{ }


\begin{document}

%%% TIKZ STUFF
\tikzset{   
        every picture/.style={remember picture,baseline},
        every node/.style={anchor=base,align=center,outer sep=1.5pt},
        every path/.style={thick},
        }
\newcommand\marktopleft[1]{%
    \tikz[overlay,remember picture] 
        \node (marker-#1-a) at (-.3em,.3em) {};%
}
\newcommand\markbottomright[2]{%
    \tikz[overlay,remember picture] 
        \node (marker-#1-b) at (0em,0em) {};%
}
\tikzstyle{every picture}+=[remember picture] 
\tikzstyle{mybox} =[draw=black, very thick, rectangle, inner sep=10pt, inner ysep=20pt]
\tikzstyle{fancytitle} =[draw=black,fill=red, text=white]
%%%% END TIKZ STUFF


\begin{frame}
\maketitle
\centering \textbf{\color{blue} Посиделка 9:}  Организация DL-экспериментов
\end{frame}


\begin{frame}{Проблемы при обучении нейросетей}
	\begin{wideitemize}
		\item  «Neural net training is a leaky abstraction» — Andrej Karpathy
		\item  Знания архитектур, оптимизаторов порой недостаточно для получения хорошей модели
		\item Универсально наилучшего решения не бывает
		\item Важна точка начала экспериментов и инкрементальные улучшения
	\end{wideitemize}
	\vspace{1.3cm}
	\vfill
	\footnotesize
	Andrej Karpathy: {\color{blue} \url{http://karpathy.github.io/2019/04/25/recipe/} } \newline Максим Рябинин: {\color{blue}\url{https://github.com/aosokin/dl_cshse_ami/blob/master/2021-fall/lectures/DL21-fall-lecture5-bestpractices.pdf} }
\end{frame}


\begin{frame}{Перед началом}
	\begin{wideitemize}
		\item  Используйте проверенные временем стандарты
		\item  Вместо своих моделей — архитектуры из популярных публикаций  (ResNet в зрении, ELMo/Transformer в текстах) и репозиториев [1,2,3]
		\item Adam со стандартным LR без расписания обойти нелегко
		\item Сложные функции потерь/аугментации лучше отложить
		\item Первые запуски на небольших датасетах, подвыборке или синтетике
	\end{wideitemize}

\vfill
\footnotesize
{\color{blue} [1] \url{https://github.com/pytorch/vision}  \newline
[2] \url{https://github.com/huggingface/transformers/}   \newline
[3] \url{https://github.com/pytorch/fairseq}} 

\end{frame}


\begin{frame}{Как искать ошибки}
	\begin{wideitemize}	
		\item Чтобы легче находить ошибки, снизьте число факторов влияния 
		
		\item  Баги могут быть как в определении и обучении модели, так и в проверке качества (даже в загрузке данных)
		
		\item В меньшем масштабе можно быстрее итерироваться и находить проблемы
		
		\item \alert{Пробуйте прогнать код на одном батче и оценить, насколько адекватные результаты вы получили:} есть ли сходимость, есть ли переобучение на валидации, адекватно ли меняются метрики 
		
		\item Визуализируйте всё, что можете: метрики, примеры работы модели
		
		\item DL-код — всё ещё код: полезно писать unit-тесты
	\end{wideitemize}
\end{frame}



\begin{frame}{Типичные ошибки: модели}
	\begin{wideitemize}
		\item Использование ad-hoc архитектур, когда не надо
		\item Использование нестабильных/сложных функций потерь вместо кросс-энтропии в классификации 
		\item Использование устаревших функций активации в глубоких сетях (сигмоида, тангенс)
		\item Плохая инициализация: нули/константы вместо Glorot/He
	\end{wideitemize}
\end{frame}


\begin{frame}{Типичные ошибки: данные}
	\begin{wideitemize}
		\item Отсутствие аугментации/использование некорректной аугментации, разные аугментации при обучении и валидации  (исключение — random crop)
		\item Если используете предобученные модели, препроцессинг данных должен быть максимально похожим 
		\item Считывать все данные сразу, используйте Dataset/DataLoader
	\end{wideitemize}
\end{frame}


\begin{frame}{Типичные ошибки: обучение}
	\begin{wideitemize}
		\item Не забывайте делать zero\_grad :)
		\item Переключайте model.train()/model.eval() в нужных фазах обучения
 
		\item Делайте чекпойнты, при них сохраняйте также параметры оптимизатора optim.state\_dict() и расписания
		\item Функция потерь должна быть максимально близка к метрике, которую вы оптимизируете 
	\end{wideitemize}
\end{frame}


% Посмотреть готовые инструменты для логов: 
%  https://www.wandb.com/
% [2] https://www.comet.ml/
% [3] https://neptune.ai/ 
\begin{frame}{Организация экспериментов}
	\begin{wideitemize}
		\item \alert{Тестируйте за один раз только одно изменение,} чтобы понимать влияние каждого фактора по отдельности 
		\item На ранних стадиях не обязательно учить до сходимости и использовать всю выборку целиком
		\item  Ведите лог всех экспериментов
		\item Примеры инструментов для лога экспериентов:  \newline 
		{\color{blue} 
		\url{https://www.wandb.com/}  \newline 
		\url{https://www.comet.ml/}  \newline 
		\url{https://neptune.ai/}  \newline 
		\url{https://dvc.org/}  \newline 
		\url{https://mlflow.org/}  \newline }
	\end{wideitemize}	
\end{frame}


\begin{frame}{Как улучшать качество?}
	\begin{wideitemize}
		\item Функция потерь должна быть максимально близка к метрике
		
		\item  Начните с небольших экспериментов и масштабируйтесь, когда всё отлажено 
		
		\item  \alert{Работа с данными (количество, качество, предобработка) зачастую приносит гораздо больше эффекта, чем перебор архитектур и оптимизаторов}
		
		\item  Архитектуры влияют существенно, но учитывайте свои ресурсы
		
		\item Занимайтесь оптимизацией гиперпараметров в самую последнюю очередь 
		
		\item Размер батча важен (ряд моделей иначе просто не учится)

	\end{wideitemize}
\end{frame}


\begin{frame}{Выводы}
	\begin{wideitemize}
		\item  Пользуйтесь проверенными техниками и опытом других людей
		
		\item  Начните с небольших экспериментов
		
		\item  Когда всё протестировано, можно масштабироваться
		
		\item  Отслеживайте все доступные метрики
		
		\item  Тестируйте одно изменение за раз
		
		\item  Сохраняйте код/конфигурацию всех экспериментов и их результаты
		
	\end{wideitemize}
\end{frame}


\end{document}

