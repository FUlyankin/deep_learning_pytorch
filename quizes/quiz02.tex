%!TEX TS-program = xelatex
\documentclass[12pt, a4paper, oneside]{article}

\input{preamble.tex}

% эпиграфы
\usepackage{epigraph}
\setlength\epigraphwidth{.8\textwidth}
\setlength\epigraphrule{0pt}

\usepackage{alltt}

\begin{document}

\section*{Quiz 2: матричные производные и градиентный спуск}

\epigraph{Во многой мудрости, много печали, и кто умножает познания, умножает скорбь.}{\textit{Царь Соломон (Екклезиаст 1:17,18)}}

\vspace{-0.5cm}
\subsection*{[4] Задание 1}
\vspace{-0.5cm}

Пусть $y_i \ge 0$. Илон Маск обучает линейную регрессию  $y = w x$ с квадратической логарифмической функцией ошибки (mean squared logarithmic error) и $l_2$ регуляризацией. Для этого он использует функцию потерь 

\[
L(w) = \frac{1}{2n} \sum_{i=1}^n (\ln(y_i + 1) - \ln(w \cdot x_i + 1))^2 +  \frac{1}{2} \cdot w^2 \to \min_{w}.
\]

Найдите значение градиента $\nabla_w L(w)$, выпишите формулу для шага градиентного спуска. Как думаете, в каких ситуациях на практике используют эту функцию ошибки? Зачем к $y_i$ прибавляют единицу? 

\vspace{-0.5cm}
\subsection*{[3] Задание 2}
\vspace{-0.5cm}
Выберите все верные утверждения про градиентный спуск и предложите, как исправить остальные пункты, чтобы они стали правильными:
\begin{enumerate}
    \item Градиентный спуск используется для оптимизации весов в линейной регрессии с $MAE,$ потому что аналитический минимум посчитать нельзя.
    \item Если взять длину шага $\eta_k = e^{-\lambda k}, \lambda > 0$, то \textbf{НЕ} гарантируется, что градиентный спуск сойдется.
    \item Для выпуклой функции нет смысла делать мульти-старт градиентного спуска.
    \item В SGD, скорее всего, потребуется больше итераций для сходимости, чем в обычном градиентном спуске.
    \item AdaGrad — это пример метода, который реализует инерцию (momentum).
    \item Чем больше размер mini-batch в SGD, тем больше дисперсия оценки градиента.
\end{enumerate}

\vspace{-0.5cm}
\subsection*{[3] Задание 3}
\vspace{-0.5cm}
Найдите $\frac{\partial y}{\partial x}$ и  $\frac{\partial y}{\partial A}$ для функции
\[
y = x^TAc , \quad A\in \RR^{n \times n}, x\in \mathbb{R}^{n}, c\in \mathbb{R}^{n} 
\]

\vspace{-0.5cm}
\subsection*{[0.1] Задание 4}
\vspace{-0.5cm}
Расскажи о своей любимой еде, почему она тебе нравится?

\end{document}