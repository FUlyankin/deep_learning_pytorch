%!TEX TS-program = xelatex
\documentclass[12pt, a4paper, oneside]{article}

\input{preamble.tex}

% эпиграфы
\usepackage{epigraph}
\setlength\epigraphwidth{.8\textwidth}
\setlength\epigraphrule{0pt}

\usepackage{alltt}

\begin{document}

\section*{Quiz 1: нейросеть --- всего лишь функция}

\epigraph{
— Меня зовут Алиса, а... \\ 
— Какое глупое имя! – нетерпеливо прервал ее Шалтай. – Что оно значит? \\
— Разве имя должно что-то значить? – проговорила Алиса с сомнением. \\ 
— Конечно, должно, — ответил Шалтай-Болтай и фыркнул. – Возьмём, к примеру, моё имя – оно выражает мою суть! Замечательную и чудесную суть! А с таким именем как, у тебя, ты можешь оказаться чем угодно. Ну просто чем угодно!
}{\textit{Льюис Кэрролл (Алиса в Зазеркалье)}}

\vspace{-0.5cm}
\subsection*{[3] Задание 1} 
\vspace{-0.5cm}

Леонардо Вильгельм Ди Каприо хочет обучить полносвязную нейронную сеть для классификации изображений. Изображения чёрно-белые, имеют размер $40 \times 40$, в обучающей выборке представлено $10$ классов. Какие значения нужно подставить вместо None в коде нейронной сети ниже? Подсчитайте общее число обучаемых параметров в сети. 

\begin{alltt}
model = nn.Sequential(
    nn.Linear(None, 100, bias=True), 
    nn.ReLU(),
    nn.Linear(100, 50, bias=False), 
    nn.ReLU(),
    nn.Linear(50, None, bias=True)
)
\end{alltt}

\vspace{-1cm}
\subsection*{[3] Задание 2}
\vspace{-0.5cm}

Джордж Тимати Клуни написал код для одного шага обучения нейронной сети, но наделал ошибок. Исправьте все его ошибки.
\begin{alltt}
    logits = model(X)
    loss = criterion(logits, labels)
    optimizer.zero\_grad()
    optimizer.step()
\end{alltt}

\vspace{-0.5cm}
\subsection*{[4] Задание 3}
\vspace{-0.5cm}
Джон Кристофер Депп раздобыл следующий набор данных:
\begin{center}
    \begin{tabular}{ccccccccc}
        \toprule
        $x_i$ & -1 & 1 & -1 & 1 & 0 & 2 & 0 & -2\\ 
        $z_i$ & -1 & -1 & 1 & 1 & 2 & 0 & -2 & 0\\ 
        $y_i$ & 0 & 0 & 0 & 0 & 1 & 1 & 1 & 1\\
        \bottomrule
    \end{tabular}
\end{center}

\begin{enumerate}
    \item[а)] Существует ли перспетрон с константой, двумя входами и пороговой функцией активации, способный идеально классифицировать $y_i$ на данной выборке? А хватит ли двух таких персептронов? А может хватит трёх?
    \item[б)] Введите такое преобразование исходных признаков $h_i = h(x_i, z_i)$, при котором с идеальной классификацией $y_i$ справился бы даже персептрон с одним входом, константой и пороговой функцией активации.
\end{enumerate}

\vspace{-0.5cm}
\subsection*{[0.1] Задание 4} 
\vspace{-0.5cm} 
\noindent-- Можно я умыкну у вас стул?  \\
-- Да. \\ 
-- Спасибо, это отрадно! \\
Такой диалог состоялся у меня в Петербурге, когда я хотел забрать стул. А потом я подумал: «умыкнуть, отрадно... что со мной не так?» А какие странные слова вы использовали в речи? 

\end{document}