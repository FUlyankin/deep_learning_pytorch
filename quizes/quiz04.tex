%!TEX TS-program = xelatex
\documentclass[12pt, a4paper, oneside]{article}

\input{preamble.tex}

% эпиграфы
\usepackage{epigraph}
\setlength\epigraphwidth{.8\textwidth}
\setlength\epigraphrule{0pt}

\usepackage{alltt}

\begin{document}

\section*{Quiz 4: свёрточные нейронные сети и эвристики}

\epigraph{All models are wrong, but some are useful.}{\textit{George E. P. Box}}

\vspace{-0.5cm}
\subsection*{[4] Задание 1}
\vspace{-0.5cm}

Архитектура AlexNet работает с картинками размера $227 \times 227 \times 3$.  Первый свёрточный слой содержит в себе $96$ свёрток размера $11 \times 11$ и параметр сдвига (stride) равный $4$. Дополнение нулями (padding) не используется. 

\begin{enumerate} 
    \item[а)]  Какого размера будет картинка, когда она пройдёт сквозь этот слой?  
    
    % (227 - 11)/(4 + 1) = 55
    % W’ = (W - F + 2P) / S + 1
    % $55 \times 55 \times 96$.
    
    \item[б)] Какое число параметров надо оценить?
    % (11*11*3)*96 = 35k 
\end{enumerate}

После первого слоя картинка попадает в слой пулинга размера $2 \times 2$. 
\begin{enumerate} 
    \item[в)]  Какого размера будет картинка, когда она пройдёт сквозь этот слой?  
    
    % (55 - 3)/2 + 1 = 27
    % Получается, что $27 \times 27 \times 96$.
    \item[г)] Какое число параметров надо оценить?
    % ноль 
\end{enumerate}

\vspace{-0.5cm}
\subsection*{[4] Задание 2}
\vspace{-0.5cm}

Алекс, Илья и Джеффри обучают нейронные сети. Объясните, чей подход правильный и почему.

\begin{itemize} 
\item Алекс использует полносвязный слой с $ReLU,$ а затем использует dropout. 
\item Илья использует полносвязный слой, затем делает dropout, а после применяет $ReLU.$
\item Джеффри использует полносвязный слой, затем делает dropout без функции активации. 
\end{itemize} 

\vspace{-0.5cm}
\subsection*{[2] Задание 3}
\vspace{-0.5cm}

Часто, при обучении свёрточных нейронных сеток, картинки нормируют на отрезок $[-1; 1].$ Объясните, зачем так поступают. Почему бы не выбрать для нормировки отрезок $[0; 1]?$


\vspace{-0.5cm}
\subsection*{[0.1] Задание 4}
\vspace{-0.5cm}

Придумайте для следующих самостоялок и домашек бонусные задания.


\end{document}