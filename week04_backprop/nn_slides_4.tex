%!TEX TS-program = xelatex
\documentclass[notes,12pt, aspectratio=169]{beamer}

\usepackage{amsmath,amsfonts,amssymb,amsthm,mathtools}  % пакеты для математики
\usepackage{minted}

\usepackage[english, russian]{babel} % выбор языка для документа
\usepackage[utf8]{inputenc} % задание utf8 кодировки исходного tex файла
\usepackage[X2,T2A]{fontenc}        % кодировка

\usepackage{fontspec}         % пакет для подгрузки шрифтов
\setmainfont{Helvetica}  % задаёт основной шрифт документа

% why do we need \newfontfamily:
% http://tex.stackexchange.com/questions/91507/
\newfontfamily{\cyrillicfonttt}{Helvetica}
\newfontfamily{\cyrillicfont}{Helvetica}
\newfontfamily{\cyrillicfontsf}{Helvetica}

\usepackage{unicode-math}     % пакет для установки математического шрифта
% \setmathfont{Neo Euler} % шрифт для математики

\usepackage{polyglossia}      % Пакет, который позволяет подгружать русские буквы
\setdefaultlanguage{russian}  % Основной язык документа
\setotherlanguage{english}    % Второстепенный язык документа

% Шрифт для кода
\setmonofont[Scale=0.85]{Monaco}
\usepackage{verbments}

\usepackage{pgfpages}
% These slides also contain speaker notes. You can print just the slides,
% just the notes, or both, depending on the setting below. Comment out the want
% you want.
%\setbeameroption{hide notes} % Only slide
%\setbeameroption{show only notes} % Only notes
%\setbeameroption{show notes on second screen=right} % Both

\usepackage{array}

\usepackage{tikz}
\usepackage{verbatim}
\setbeamertemplate{note page}{\pagecolor{yellow!5}\insertnote}
\usetikzlibrary{positioning}
\usetikzlibrary{snakes}
\usetikzlibrary{calc}
\usetikzlibrary{arrows}
\usetikzlibrary{decorations.markings}
\usetikzlibrary{shapes.misc}
\usetikzlibrary{matrix,shapes,arrows,fit,tikzmark}

\usepackage{hyperref}
\usepackage{lipsum}
\usepackage{multimedia}
\usepackage{multirow}
\usepackage{dcolumn}
\usepackage{bbm}
\newcolumntype{d}[0]{D{.}{.}{5}}

\usepackage{changepage}
\usepackage{appendixnumberbeamer}
\newcommand{\beginbackup}{
   \newcounter{framenumbervorappendix}
   \setcounter{framenumbervorappendix}{\value{framenumber}}
   \setbeamertemplate{footline}
   {
     \leavevmode%
     \hline
     box{%
       \begin{beamercolorbox}[wd=\paperwidth,ht=2.25ex,dp=1ex,right]{footlinecolor}%
%         \insertframenumber  \hspace*{2ex} 
       \end{beamercolorbox}}%
     \vskip0pt%
   }
 }
\newcommand{\backupend}{
   \addtocounter{framenumbervorappendix}{-\value{framenumber}}
   \addtocounter{framenumber}{\value{framenumbervorappendix}} 
}

% для имитации питоновского синтаксиса 
\newcommand{\pgr}[1]{{\color{green} \textbf{#1}}}


%%%%%%%%%% Работа с картинками %%%%%%%%%
\usepackage{graphicx}                  % Для вставки рисунков
\usepackage{graphics}
\graphicspath{{images/}}    % можно указать папки с картинками
\usepackage{wrapfig}                   % Обтекание рисунков и таблиц текстом

\usepackage[space]{grffile}
\usepackage{booktabs}

% These are my colors -- there are many like them, but these ones are mine.
\definecolor{blue}{RGB}{0,114,178}
\definecolor{red}{RGB}{213,94,0}
\definecolor{yellow}{RGB}{240,228,66}
\definecolor{green}{RGB}{0,128, 0}


\definecolor{amethyst}{rgb}{0.6, 0.4, 0.8}
\definecolor{junglegreen}{rgb}{0.16, 0.67, 0.53}

\hypersetup{
  colorlinks=true,
  linkbordercolor = {white},
  linkcolor = {blue},
  urlcolor= {blue}
}


%% I use a beige off white for my background
\definecolor{MyBackground}{RGB}{255,253,218}

%% Uncomment this if you want to change the background color to something else
%\setbeamercolor{background canvas}{bg=MyBackground}

%% Change the bg color to adjust your transition slide background color!
\newenvironment{transitionframe}{
  \setbeamercolor{background canvas}{bg=yellow}
  \begin{frame}}{
    \end{frame}
}

\setbeamercolor{frametitle}{fg=blue}
\setbeamercolor{title}{fg=black}
\setbeamertemplate{footline}[frame number]
\setbeamertemplate{navigation symbols}{} 
\setbeamertemplate{itemize items}{-}
\setbeamercolor{itemize item}{fg=blue}
\setbeamercolor{itemize subitem}{fg=blue}
\setbeamercolor{enumerate item}{fg=blue}
\setbeamercolor{enumerate subitem}{fg=blue}
\setbeamercolor{button}{bg=MyBackground,fg=blue,}


% If you like road maps, rather than having clutter at the top, have a roadmap show up at the end of each section 
% (and after your introduction)
% Uncomment this is if you want the roadmap!
% \AtBeginSection[]
% {
%    \begin{frame}
%        \frametitle{Roadmap of Talk}
%        \tableofcontents[currentsection]
%    \end{frame}
% }
\setbeamercolor{section in toc}{fg=blue}
\setbeamercolor{subsection in toc}{fg=red}
\setbeamersize{text margin left=1em,text margin right=1em} 

% списки, которые растягиваются на всю величину слайда 
\newenvironment{wideitemize}{\itemize\addtolength{\itemsep}{10pt}}{\enditemize}


\usepackage{xcolor}

% Syntax: \colorboxed[<color model>]{<color specification>}{<math formula>}
\newcommand*{\colorboxed}{}
\def\colorboxed#1#{%
	\colorboxedAux{#1}%
}

\newcommand*{\colorboxedAux}[3]{%
	% #1: optional argument for color model
	% #2: color specification
	% #3: formula
	\begingroup
	\colorlet{cb@saved}{.}%
	\color#1{#2}%
	\boxed{%
		\color{cb@saved}%
		#3%
	}%
	\endgroup
}

\usepackage{pgfplots}
\usepackage{tikz}

\DeclareMathOperator{\logloss}{logloss}

\title[]{\textcolor{blue}{Глубокое обучение и вообще}}
\author{Ульянкин Филипп}
\date{ } 

\usepackage{ulem}

\begin{document}

%%% TIKZ STUFF
\tikzset{   
        every picture/.style={remember picture,baseline},
        every node/.style={anchor=base,align=center,outer sep=1.5pt},
        every path/.style={thick},
        }
\newcommand\marktopleft[1]{%
    \tikz[overlay,remember picture] 
        \node (marker-#1-a) at (-.3em,.3em) {};%
}
\newcommand\markbottomright[2]{%
    \tikz[overlay,remember picture] 
        \node (marker-#1-b) at (0em,0em) {};%
}
\tikzstyle{every picture}+=[remember picture] 
\tikzstyle{mybox} =[draw=black, very thick, rectangle, inner sep=10pt, inner ysep=20pt]
\tikzstyle{fancytitle} =[draw=black,fill=red, text=white]
%%%% END TIKZ STUFF

% Title Slide

\begin{frame}
\maketitle
\centering \textbf{\color{blue} Посиделка 4:}  Алгоритм обратного распространения ошибки
\end{frame}

\begin{frame}{Agenda}
\begin{wideitemize}
	\item Алгоритм обратного распространения ошибки 
\end{wideitemize} 
\end{frame}


\begin{transitionframe}
	\begin{center}
		{\Huge Как обучить нейросеть?} \\ \mbox{ } \\
			\begin{tikzpicture}
			\node[inner sep=0pt] (russell) at (0,0)
			{\includegraphics[scale = 0.15]{ml.png}};
			\node[] at (0,-0.6) {Ты необучаем!};
			\end{tikzpicture}
	\end{center}
\end{transitionframe}


\begin{frame}{Нейросеть —  сложная функция}
	\begin{wideitemize}
	\item Прямое распространение ошибки (forward propagation): 
	
	\[ X \Rightarrow X \cdot W_1 \Rightarrow f(X \cdot W_1) \Rightarrow f(X \cdot W_1) \cdot W_2 \Rightarrow \ldots \Rightarrow \hat{y} \]
	
	\item Считаем потери:
	
	\[Loss = \frac{1}{2} (y - \hat y)^2\]
	
	\item Все слои обычно дифференцируемы, поэтому можно посчитать производные по всем параметрам
	
	\item Для обучения можно использовать градиентный спуск
	\end{wideitemize}
\end{frame}


\begin{frame}{Как обучить нейросеть?}

\[ L(W_1, W_2) =  \frac{1}{2} \cdot (y - f(X \cdot W_1) \cdot W_2)^2\]

\begin{center}
\alert{Секрет успеха в умении брать производную}
\end{center}

\pause

\[ \boxed{ f(g(x))' = f'(g(x)) \cdot g'(x) }  \]

\pause

\begin{equation*} 
\begin{aligned} 
\frac{\partial L}{\partial W_2} &=   { \only<2>{ \color{red}} - (y - f(X \cdot W_1) \cdot W_2) } \cdot f(X \cdot W_1) \\
\frac{\partial L}{\partial W_1} &= { \only<2>{ \color{red}}  - (y - f(X \cdot W_1) \cdot W_2) } \cdot W_2 \cdot  f'(X \cdot W_1) \cdot X 
\end{aligned}
\end{equation*}

\vfill

\pause

\alert{Дважды ищем одно и то же $\Rightarrow$ оптимизация поиска производных даст нам алгоритм обратного распространения ошибки (back-propagation)}
\end{frame}


\begin{frame}{Back-propagation}
	\begin{center}
		\includegraphics[width=.8\linewidth]{backpropagation.png}
	\end{center}
\end{frame}


\begin{frame}{Цепное правило}
	\begin{wideitemize}
		\item  Возьмём сложную функцию: 
		
		\begin{equation*}
		\begin{aligned}
		& z_1 = z_1(x_1, x_2) \\  & z_2 = z_2(x_1, x_2) \\  & y = y(z_1, z_2)
		\end{aligned}
		\end{equation*} 
		
		\item Производную такой функции можно найти по цепному правилу: 
		
		$$
		\frac{\partial y}{\partial x_1} = {\color{amethyst} \frac{\partial y}{\partial z_1}} \cdot {\color{red} \frac{\partial z_1}{\partial x_1}} + {\color{junglegreen} \frac{\partial y}{\partial z_2}} \cdot {\color{green} \frac{\partial z_2}{\partial x_1}}
		$$ 
	\end{wideitemize}
\end{frame} 


\begin{frame}{Как считать производные?}
\begin{columns}
	\begin{column}{0.6\textwidth}
\begin{center}
	\begin{tikzpicture}
	\tikzstyle{place}=[circle, draw=black, minimum size = 8mm]
	
	% Input
	\foreach \x in {1,...,2}
	\draw node at (0, -\x*1.5) [place] (first_\x) {$x_\x$};
	
	% Hidden 1
	\foreach \x in {1,...,2}
	\node at (3, -\x*1.5) [place] (second_\x){$z_\x$};		
		
	% Output
	\node at (6, -2.25) [place] (fourth){$y$};
	
	\draw [->, red]  (first_1) to (second_1);
	\draw [->, green]  (first_1) to (second_2);
	\draw [->]  (first_2) to (second_1);
	\draw [->]  (first_2) to (second_2);
	
	\draw [->, amethyst]  (second_1) to (fourth);
	\draw [->, junglegreen]  (second_2) to (fourth);
	\end{tikzpicture}
	\vfill
	\begin{tikzpicture}
	\tikzstyle{place}=[circle, draw=black, minimum size = 8mm]
	
	% Input
	\foreach \x in {1,...,2}
	\draw node at (0, -\x*1.5) [place] (second_\x) {$x_\x$};
	
	% Hidden 1
	\foreach \x in {1,...,2}
	\node at (3, -\x*1.5) [place] (first_\x){$z_\x$};		
	
	% Output
	\node at (6, -2.25) [place] (fourth){$y$};
	
	\draw [->, dashed, red]  (first_1) to (second_1) node[right=1.cm] {$\frac{\partial z_1}{\partial x_1}$} ;
	\draw [->, dashed]  (first_1) to (second_2) node[right=1.cm,above] {$\frac{\partial z_1}{\partial x_2}$} ;
	\draw [->, dashed, green]  (first_2) to (second_1) node[right=2.3cm, below] {$\frac{\partial z_2}{\partial x_1}$} ;
	\draw [->, dashed]  (first_2) to (second_2) node[right=1.2cm] {$\frac{\partial z_2}{\partial x_2}$} ;
	
	\draw [->, dashed, amethyst] (fourth)  to (first_1) node[right=1.cm] {$\frac{\partial y}{\partial z_1}$} ;
	\draw [->, dashed, junglegreen] (fourth) to (first_2) node[right=1.cm] {$\frac{\partial y}{\partial z_2}$} ;
	\end{tikzpicture}
\end{center}
\end{column}
\begin{column}{0.4\textwidth}
\textbf{Граф вычислений: }
\begin{equation*} 
\begin{aligned}
	& z_1 = z_1(x_1, x_2)\\
	& z_2 = z_2(x_1, x_2) \\
	& y = y(z_1, z_2) 
\end{aligned}
\end{equation*} 
\vfill 
\only<1>{
\alert{Из него можно построить граф производных, каждому ребру будет приписана производная}}

\only<2>{\textbf{Можно догадаться как работает цепное правило:}
	\[\frac{\partial y}{\partial x_1} = {\color{amethyst} \frac{\partial y}{\partial z_1}} \cdot {\color{red} \frac{\partial z_1}{\partial x_1}} + {\color{junglegreen} \frac{\partial y}{\partial z_2}} \cdot {\color{green} \frac{\partial z_2}{\partial x_1}} \]
}
\end{column}
\end{columns}
\end{frame} 



\begin{frame}{Пойдём глубже}

\begin{center}
	\begin{tikzpicture}
	\tikzstyle{place}=[circle, draw=black, minimum size = 8mm]
	
	% Input
	\foreach \x in {1,...,2}
	\draw node at (0, -\x*1.5) [place] (first_\x) {$x_\x$};
	
	% Hidden 1
	\foreach \x in {1,...,2}
	\node at (3, -\x*1.5) [place] (second_\x){$z_\x$};		
	
	% Hidden 2
	\foreach \x in {1,...,2}
	\node at (6, -\x*1.5) [place] (third_\x){$h_\x$};	
	
	% Output
	\node at (9, -2.25) [place] (fourth){$y$};
	
	\draw [->, red]  (first_1) to (second_1);
	\draw [->, green]  (first_1) to (second_2);
	\draw [->, red]  (first_2) to (second_1);
	\draw [->, green]  (first_2) to (second_2);
	
	\draw [->, amethyst]  (second_1) to (third_1);
	\draw [->, junglegreen]  (second_1) to (third_2);
	\draw [->, amethyst]  (second_2) to (third_1);
	\draw [->, junglegreen]  (second_2) to (third_2);
	
	\draw [->, blue]  (third_1) to (fourth);
	\draw [->, blue]  (third_2) to (fourth);

	\end{tikzpicture}
\end{center}

\only<1>{
\begin{equation*} 
	\begin{aligned}
		z_1 = z_1({\color{red} x_1, x_2})  & \quad   h_1 = h_1({\color{amethyst} z_1, z_2}) & \quad y = y({\color{blue} h_1, h_2}) \\
		z_2 = z_2({\color{green} x_1, x_2}) & \quad h_2 = h_2({\color{junglegreen} z_1, z_2}) &  
	\end{aligned}
\end{equation*} }

\only<2>{
\[ \frac{\partial y}{\partial x_1} = ?  \]	
}

\only<3>{
	\[ \frac{\partial y}{\partial x_1} = {\color{blue} \frac{\partial y}{\partial h_1}} \cdot  \frac{\partial h_1}{\partial x_1} + {\color{blue} \frac{\partial y}{\partial h_2}} \cdot \frac{\partial h_2}{\partial x_1}   \]	
}

\only<4>{
	\[ \frac{\partial y}{\partial x_1} = {\color{blue} \frac{\partial y}{\partial h_1}} \cdot  \boxed{ \frac{\partial h_1}{\partial x_1}  } + {\color{blue} \frac{\partial y}{\partial h_2}} \cdot \boxed{ \frac{\partial h_2}{\partial x_1} }  \]	
}

\only<5>{
	\[ \frac{\partial y}{\partial x_1} = {\color{blue} \frac{\partial y}{\partial h_1}} \cdot \underbrace{ \boxed{ \frac{\partial h_1}{\partial x_1}  }  }_{ {\color{amethyst} \frac{\partial h_1}{\partial z_1}} \cdot {\color{red} \frac{\partial z_1}{\partial x_1}} + {\color{amethyst} \frac{\partial h_1}{\partial z_2}} \cdot {\color{green} \frac{\partial z_2}{\partial x_1}}  }  + {\color{blue} \frac{\partial y}{\partial h_2}} \cdot \underbrace{\boxed{ \frac{\partial h_2}{\partial x_1} } }_{ {\color{junglegreen} \frac{\partial h_2}{\partial z_1}} \cdot {\color{red} \frac{\partial z_1}{\partial x_1}} + {\color{junglegreen} \frac{\partial h_2}{\partial z_2}} \cdot {\color{green} \frac{\partial z_2}{\partial x_1}}  } \]	
}


\only<6>{
	\[ \frac{\partial y}{\partial x_1} = {\color{blue} \frac{\partial y}{\partial h_1}} \cdot \left( {\color{amethyst} \frac{\partial h_1}{\partial z_1}} \cdot {\color{red} \frac{\partial z_1}{\partial x_1}} + {\color{amethyst} \frac{\partial h_1}{\partial z_2}} \cdot {\color{green} \frac{\partial z_2}{\partial x_1}}  \right)  + {\color{blue} \frac{\partial y}{\partial h_2}} \cdot \left( {\color{junglegreen} \frac{\partial h_2}{\partial z_1}} \cdot {\color{red} \frac{\partial z_1}{\partial x_1}} + {\color{junglegreen} \frac{\partial h_2}{\partial z_2}} \cdot {\color{green} \frac{\partial z_2}{\partial x_1}}  \right) \]	
}


\only<7>{
	\[ \frac{\partial y}{\partial x_1} = {\color{blue} \frac{\partial y}{\partial h_1}}  {\color{amethyst} \frac{\partial h_1}{\partial z_1}}  {\color{red} \frac{\partial z_1}{\partial x_1}}    +	{\color{blue} \frac{\partial y}{\partial h_1}}  {\color{amethyst} \frac{\partial h_1}{\partial z_2}}  {\color{green} \frac{\partial z_2}{\partial x_1}}  + {\color{blue} \frac{\partial y}{\partial h_2}}   {\color{junglegreen} \frac{\partial h_2}{\partial z_1}}  {\color{red} \frac{\partial z_1}{\partial x_1}} + {\color{blue} \frac{\partial y}{\partial h_2}}  {\color{junglegreen} \frac{\partial h_2}{\partial z_2}}  {\color{green} \frac{\partial z_2}{\partial x_1}}  \]	
}

\end{frame} 


\begin{frame}{Пойдём глубже}

\begin{center}
	\begin{tikzpicture}
	\tikzstyle{place}=[circle, draw=black, minimum size = 8mm]
	
	% Input
	\foreach \x in {1,...,2}
	\draw node at (0, -\x*1.5) [place] (first_\x) {$x_\x$};
	
	% Hidden 1
	\foreach \x in {1,...,2}
	\node at (3, -\x*1.5) [place] (second_\x){$z_\x$};		
	
	% Hidden 2
	\foreach \x in {1,...,2}
	\node at (6, -\x*1.5) [place] (third_\x){$h_\x$};	
	
	% Output
	\node at (9, -2.25) [place] (fourth){$y$};
		
	\draw [->, dashed, red]  (second_1) to (first_1);
	\draw [->, dashed, green]  (second_2) to (first_1);
	\draw [->, dashed, red]  (second_1) to (first_2);
	\draw [->, dashed, green]  (second_2) to (first_2);
	
	\draw [->, dashed, amethyst]  (third_1) to (second_1);
	\draw [->, dashed, junglegreen]  (third_2) to (second_1);
	\draw [->, dashed, amethyst] (third_1) to (second_2);
	\draw [->, dashed, junglegreen]  (third_2) to (second_2);
	
	\draw [->, dashed,  blue]  (fourth) to (third_1);
	\draw [->, dashed, blue]   (fourth) to (third_2);
	
	\only<2>{ 
			\draw [->, dashed, ultra thick, blue]  (fourth) to (third_1);
			\draw [->, dashed, ultra thick, amethyst]  (third_1) to (second_1);
			\draw [->, dashed, ultra thick, red]  (second_1) to (first_1);
	}

	\only<3>{ 
		\draw [->, dashed, ultra thick, blue]  (fourth) to (third_1);
		\draw [->, dashed, ultra thick, amethyst]  (third_1) to (second_2);
		\draw [->, dashed, ultra thick, green]  (second_2) to (first_1);
	}

	\only<4>{ 
		\draw [->, dashed, ultra thick, blue]  (fourth) to (third_2);
		\draw [->, dashed, ultra thick, junglegreen]  (third_2) to (second_1);
		\draw [->, dashed, ultra thick, red]  (second_1) to (first_1);
	}

	\only<5>{ 
		\draw [->, dashed, ultra thick, blue]  (fourth) to (third_2);
		\draw [->, dashed, ultra thick, junglegreen]  (third_2) to (second_2);
		\draw [->, dashed, ultra thick, green]  (second_2) to (first_1);
	}
	
	\end{tikzpicture}
\end{center}

\only<1>{
	\[ 
	\frac{\partial y}{\partial x_1} = {\color{blue} \frac{\partial y}{\partial h_1}}  {\color{amethyst} \frac{\partial h_1}{\partial z_1}}  {\color{red} \frac{\partial z_1}{\partial x_1}}    +     {\color{blue} \frac{\partial y}{\partial h_1}}  {\color{amethyst} \frac{\partial h_1}{\partial z_2}}  {\color{green} \frac{\partial z_2}{\partial x_1}}     +       {\color{blue} \frac{\partial y}{\partial h_2}}   {\color{junglegreen} \frac{\partial h_2}{\partial z_1}}  {\color{red} \frac{\partial z_1}{\partial x_1}}      +      {\color{blue} \frac{\partial y}{\partial h_2}}  {\color{junglegreen} \frac{\partial h_2}{\partial z_2}}  {\color{green} \frac{\partial z_2}{\partial x_1}}  
	\]	
}

\only<2>{
	\[ 
	\frac{\partial y}{\partial x_1} =  \boxed{ {\color{blue} \frac{\partial y}{\partial h_1}}  {\color{amethyst} \frac{\partial h_1}{\partial z_1}}  {\color{red} \frac{\partial z_1}{\partial x_1}} }   +     {\color{blue} \frac{\partial y}{\partial h_1}}  {\color{amethyst} \frac{\partial h_1}{\partial z_2}}  {\color{green} \frac{\partial z_2}{\partial x_1}}     +       {\color{blue} \frac{\partial y}{\partial h_2}}   {\color{junglegreen} \frac{\partial h_2}{\partial z_1}}  {\color{red} \frac{\partial z_1}{\partial x_1}}      +      {\color{blue} \frac{\partial y}{\partial h_2}}  {\color{junglegreen} \frac{\partial h_2}{\partial z_2}}  {\color{green} \frac{\partial z_2}{\partial x_1}}  
	\]	
}


\only<3>{
	\[ 
	\frac{\partial y}{\partial x_1} = {\color{blue} \frac{\partial y}{\partial h_1}}  {\color{amethyst} \frac{\partial h_1}{\partial z_1}}  {\color{red} \frac{\partial z_1}{\partial x_1}}    +    \boxed{ {\color{blue} \frac{\partial y}{\partial h_1}}  {\color{amethyst} \frac{\partial h_1}{\partial z_2}}  {\color{green} \frac{\partial z_2}{\partial x_1}} }    +       {\color{blue} \frac{\partial y}{\partial h_2}}   {\color{junglegreen} \frac{\partial h_2}{\partial z_1}}  {\color{red} \frac{\partial z_1}{\partial x_1}}      +      {\color{blue} \frac{\partial y}{\partial h_2}}  {\color{junglegreen} \frac{\partial h_2}{\partial z_2}}  {\color{green} \frac{\partial z_2}{\partial x_1}}  
	\]	
}

\only<4>{
	\[ 
	\frac{\partial y}{\partial x_1} = {\color{blue} \frac{\partial y}{\partial h_1}}  {\color{amethyst} \frac{\partial h_1}{\partial z_1}}  {\color{red} \frac{\partial z_1}{\partial x_1}}    +     {\color{blue} \frac{\partial y}{\partial h_1}}  {\color{amethyst} \frac{\partial h_1}{\partial z_2}}  {\color{green} \frac{\partial z_2}{\partial x_1}}     +      \boxed{ {\color{blue} \frac{\partial y}{\partial h_2}}   {\color{junglegreen} \frac{\partial h_2}{\partial z_1}}  {\color{red} \frac{\partial z_1}{\partial x_1}} }     +      {\color{blue} \frac{\partial y}{\partial h_2}}  {\color{junglegreen} \frac{\partial h_2}{\partial z_2}}  {\color{green} \frac{\partial z_2}{\partial x_1}}  
	\]	
}

\only<5>{
	\[ 
	\frac{\partial y}{\partial x_1} = {\color{blue} \frac{\partial y}{\partial h_1}}  {\color{amethyst} \frac{\partial h_1}{\partial z_1}}  {\color{red} \frac{\partial z_1}{\partial x_1}}    +     {\color{blue} \frac{\partial y}{\partial h_1}}  {\color{amethyst} \frac{\partial h_1}{\partial z_2}}  {\color{green} \frac{\partial z_2}{\partial x_1}}     +       {\color{blue} \frac{\partial y}{\partial h_2}}   {\color{junglegreen} \frac{\partial h_2}{\partial z_1}}  {\color{red} \frac{\partial z_1}{\partial x_1}}      +    \boxed{  {\color{blue} \frac{\partial y}{\partial h_2}}  {\color{junglegreen} \frac{\partial h_2}{\partial z_2}}  {\color{green} \frac{\partial z_2}{\partial x_1}}  }
	\]	
}

\end{frame} 


\begin{frame}{Алгоритм поиска производной в графе}
	\begin{wideitemize}
		\item Как посчитать производную $a$ по $b$? 
		
		\item Находим непосещённый путь из $a$ в $b$ 
		
		\item Перемножаем значения на рёбрах пути 
		
		\item Добавляем в сумму
		
	\end{wideitemize}	
\vspace{-1.5cm}
\begin{columns}
\begin{column}{0.4\textwidth}
\begin{center}
\begin{tikzpicture}
\tikzstyle{place}=[circle, draw=black, minimum size = 8mm]

% Input
\foreach \x in {1,...,2}
\draw node at (0, -\x*1.5) [place] (second_\x) {$x_\x$};

% Hidden 1
\foreach \x in {1,...,2}
\node at (3, -\x*1.5) [place] (first_\x){$z_\x$};		

% Output
\node at (6, -2.25) [place] (fourth){$y$};

\draw [->, dashed, red]  (first_1) to (second_1) node[right=1.cm] {$\frac{\partial z_1}{\partial x_1}$} ;
\draw [->, dashed]  (first_1) to (second_2);
\draw [->, dashed, green]  (first_2) to (second_1) node[right=2.3cm, below] {$\frac{\partial z_2}{\partial x_1}$} ;
\draw [->, dashed]  (first_2) to (second_2); 

\draw [->, dashed, amethyst] (fourth)  to (first_1) node[right=1.cm] {$\frac{\partial y}{\partial z_1}$} ;
\draw [->, dashed, junglegreen] (fourth) to (first_2) node[right=1.cm] {$\frac{\partial y}{\partial z_2}$} ;
\end{tikzpicture}
\end{center}
\end{column}
\begin{column}{0.4\textwidth}
\[ 
\frac{\partial y}{\partial x_1} =  {\color{amethyst} \frac{\partial y}{\partial z_1} } \cdot {\color{red}  \frac{\partial z_1}{\partial x_1} } + {\color{ junglegreen}  \frac{\partial y}{\partial z_2} } \cdot {\color{green} \frac{\partial z_2}{\partial x_1 } }
\]
\end{column}
\end{columns}
\end{frame}


\begin{frame}{На примере одного нейрона}
\begin{center}
	\begin{tikzpicture}
		\tikzstyle{place}=[circle, draw=black, minimum size = 8mm]
		
		% Input
		\foreach \x in {1,...,2}
		\draw node at (0, -\x*1.5) [place] (first_\x) {$x_\x$};
		
		% weights
		\draw node at (1, -0.5) [place] (param_1) {$w_1$};
		\draw node at (1, -4) [place] (param_2) {$w_2$};
		
		% product
		\foreach \x in {1,...,2}
		\node at (3, -\x*1.5) [place] (second_\x){$\cdot$};		

		% sum
		\node at (5, -2.25) [place] (sum){$s$};
		\node at (7, -2.25) [place] (activ){$f$};
		\node at (9, -2.25) [place] (out){$z$};
		\node at (11, -2.75) [place] (loss){$L$};
		\node at (9, -3.25) [place] (target){$y$};
		
		\draw [->]  (first_1) to (second_1);
		\draw [->]  (first_2) to (second_2);	
		\draw [->]  (param_1) to (second_1);
		\draw [->]  (param_2) to (second_2);
		
		\draw [->] (second_1) to (sum);
		\draw [->] (second_2) to (sum);
		
		\draw [->] (sum) to (activ);
		\draw [->] (activ) to (out);
		\draw [->] (out) to (loss);
		\draw [->] (target) to (loss);
	\end{tikzpicture}

	
	\begin{equation*}
			z = f(s) = f(w_1 \cdot x_1 + w_2 \cdot x_2)
	\end{equation*}
	
	\alert{ Для SGD нам нужны $\frac{\partial L}{\partial w_1}$  и $\frac{\partial L}{\partial w_2}$ }
\end{center}
\end{frame}	



\begin{frame}{Граф производных}
\begin{center}
	\begin{tikzpicture}
	\tikzstyle{place}=[circle, draw=black, minimum size = 8mm]
	
	% Input
	\foreach \x in {1,...,2}
	\draw node at (0, -\x*1.5) [place] (first_\x) {$x_\x$};
	
	% weights
	\draw node at (1, -0.5) [place] (param_1) {$w_1$};
	\draw node at (1, -4) [place] (param_2) {$w_2$};
	
	% product
	\foreach \x in {1,...,2}
	\node at (3, -\x*1.5) [place] (second_\x){$\cdot$};		
	
	% sum
	\node at (5, -2.25) [place] (sum){$s$};
	\node at (7, -2.25) [place] (activ){$f$};
	\node at (9, -2.25) [place] (out){$z$};
	\node at (11, -2.75) [place] (loss){$L$};
	\node at (9, -3.25) [place] (target){$y$};
	
	\only<1>{
	\draw [->, dashed, red]  (second_1) to (first_1) node[below right=2.mm] {$w_1$} ;
	\draw [->, dashed, red]  (second_2) to (first_2) node[above right=2.mm] {$w_2$} ;
	\draw [->, dashed, red]  (second_1) to (param_1) node[right=1.cm] {$x_1$} ;
	\draw [->, dashed, red]  (second_2) to (param_2) node[right=1.cm] {$x_2$} ;
	
	\draw [->, dashed, red] (sum) to (second_1)  node[right=8.mm] {$1$} ;
	\draw [->, dashed, red] (sum) to (second_2)  node[right=8.mm] {$1$} ;
	
	\draw [->, dashed, red] (activ) to (sum)  node[right=6.mm] {$\frac{\partial f}{\partial s}$} ;
	\draw [->, dashed, red] (out) to (activ)  node[right=7.mm] {$1$} ;
	\draw [->, dashed] (loss) to (out) node[right=7.mm] {$\frac{\partial L}{\partial z}$} ;
	\draw [->, dashed] (loss) to (target) ;
	}

	\only<2>{
	\draw [->, dashed, red]  (second_1) to (first_1) node[below right=2.mm] {$w_1$} ;
	\draw [->, dashed, red]  (second_2) to (first_2) node[above right=2.mm] {$w_2$} ;
	\draw [->, dashed, amethyst, ultra thick]  (second_1) to (param_1) node[right=1.cm] {$x_1$} ;
	\draw [->, dashed, red]  (second_2) to (param_2) node[right=1.cm] {$x_2$} ;
	
	\draw [->, dashed, amethyst, ultra thick] (sum) to (second_1)  node[right=8.mm] {$1$} ;
	\draw [->, dashed, red] (sum) to (second_2)  node[right=8.mm] {$1$} ;
	
	\draw [->, dashed, amethyst, ultra thick] (activ) to (sum)  node[right=6.mm] {$\frac{\partial f}{\partial s}$} ;
	\draw [->, dashed, amethyst, ultra thick] (out) to (activ)  node[right=7.mm] {$1$} ;
	\draw [->, dashed, amethyst, ultra thick] (loss) to (out) node[right=7.mm] {$\frac{\partial L}{\partial z}$} ;
	\draw [->, dashed] (loss) to (target) ;
	}

	\only<3>{
	\draw [->, dashed, red]  (second_1) to (first_1) node[below right=2.mm] {$w_1$} ;
	\draw [->, dashed, red]  (second_2) to (first_2) node[above right=2.mm] {$w_2$} ;
	\draw [->, dashed, red]  (second_1) to (param_1) node[right=1.cm] {$x_1$} ;
	\draw [->, dashed, amethyst,  ultra thick]  (second_2) to (param_2) node[right=1.cm] {$x_2$} ;
	
	\draw [->, dashed, red] (sum) to (second_1)  node[right=8.mm] {$1$} ;
	\draw [->, dashed, amethyst, ultra thick] (sum) to (second_2)  node[right=8.mm] {$1$} ;
	
	\draw [->, dashed, amethyst, ultra thick] (activ) to (sum)  node[right=6.mm] {$\frac{\partial f}{\partial s}$} ;
	\draw [->, dashed, amethyst, ultra thick] (out) to (activ)  node[right=7.mm] {$1$} ;
	\draw [->, dashed, amethyst, ultra thick] (loss) to (out) node[right=7.mm] {$\frac{\partial L}{\partial z}$} ;
	\draw [->, dashed] (loss) to (target) ;
	}

	\end{tikzpicture}

	\begin{equation*}
	z = f(s) = f(w_1 \cdot x_1 + w_2 \cdot x_2)
	\end{equation*}
	
	\only<1> { \alert{ Для SGD нам нужны $\frac{\partial L}{\partial w_1}$  и $\frac{\partial L}{\partial w_2}$ } }
	
	\only<2>{ 
	\[
		\frac{\partial L}{\partial w_1} = \frac{\partial L}{\partial z} \cdot \frac{\partial f}{\partial s} \cdot x_1 
	\] }

	\only<3>{ 
	\[
		\frac{\partial L}{\partial w_1} = \frac{\partial L}{\partial z} \cdot \frac{\partial f}{\partial s} \cdot x_1  \qquad  \frac{\partial L}{\partial w_2} = \frac{\partial L}{\partial z} \cdot \frac{\partial f}{\partial s} \cdot x_2  
	\] }
\end{center}
\end{frame}	


\begin{frame}{Цепное правило и грaф производных}
\begin{wideitemize} 
	\item Теперь у нас есть алгоритм для подсчета производных для любых
	дифференцируемых графов вычислений
	
	\item \alert{ Осталось делать вычисления быстро }
\end{wideitemize} 
\end{frame}



\begin{frame}{Обратное распространение ошибки}

\alert{Мы хотим поменять параметры нейрона в рамках SGD}

\vfill

\[
h_2 = f(w_0 + w_1 z_1 + w_2 z_2)
\]

\vfill

\[
\frac{\partial L}{\partial w_1} = \frac{\partial L}{\partial y} \cdot \frac{\partial y}{\partial w_1} =  \frac{\partial L}{\partial y} \cdot {\color{blue} \frac{\partial y}{\partial h_2}} \cdot \frac{\partial h_2}{\partial w_1} 
\]

\vfill

\[
w_1^t = w_1^{t-1} - \gamma \cdot \frac{\partial L}{\partial w_1} (w_1^{t-1})
\]
\end{frame}


\begin{frame}{Обратное распространение ошибки}
\begin{equation*} 
\begin{aligned}
	\only<1-6>{ & 3: \quad  {\color{blue} \frac{\partial y}{\partial h_2} \qquad \frac{\partial y}{\partial h_1}} \\} 
	\only<2-6>{& 2: \quad \frac{\partial y}{\partial z_1} = {\color{blue} \frac{\partial y}{\partial h_1}} \cdot {\color{amethyst} \frac{\partial h_1}{\partial z_1}} + \frac{\partial y}{\partial h_2} \cdot {\color{junglegreen} \frac{\partial h_2}{\partial z_1}}   \qquad  \frac{\partial y}{\partial z_2} =  {\color{blue} \frac{\partial y}{\partial h_1}} \cdot {\color{amethyst} \frac{\partial h_1}{\partial z_2}}  + {\color{blue} \frac{\partial y}{\partial h_2}} \cdot {\color{junglegreen} \frac{\partial h_2}{\partial z_2}} \\} 
	\only<3>{& 1: \quad \frac{\partial y}{\partial x_1} = {\color{blue} \frac{\partial y}{\partial h_1}}  {\color{amethyst} \frac{\partial h_1}{\partial z_1}}  {\color{red} \frac{\partial z_1}{\partial x_1}}    +    {\color{blue} \frac{\partial y}{\partial h_2}}   {\color{junglegreen} \frac{\partial h_2}{\partial z_1}}  {\color{red} \frac{\partial z_1}{\partial x_1}}      +  {\color{blue} \frac{\partial y}{\partial h_1}}  {\color{amethyst} \frac{\partial h_1}{\partial z_2}}  {\color{green} \frac{\partial z_2}{\partial x_1}}     +             {\color{blue} \frac{\partial y}{\partial h_2}}  {\color{junglegreen} \frac{\partial h_2}{\partial z_2}}  {\color{green} \frac{\partial z_2}{\partial x_1}} \\ }  
	\only<4>{& 1: \quad \frac{\partial y}{\partial x_1} = \left( {\color{blue} \frac{\partial y}{\partial h_1}}  {\color{amethyst} \frac{\partial h_1}{\partial z_1}}    + {\color{blue} \frac{\partial y}{\partial h_2}}   {\color{junglegreen} \frac{\partial h_2}{\partial z_1}}  \right) \cdot {\color{red} \frac{\partial z_1}{\partial x_1}}  + \left( {\color{blue} \frac{\partial y}{\partial h_1}}  {\color{amethyst} \frac{\partial h_1}{\partial z_2}} +    {\color{blue} \frac{\partial y}{\partial h_2}}  {\color{junglegreen} \frac{\partial h_2}{\partial z_2}}  \right) \cdot {\color{green} \frac{\partial z_2}{\partial x_1}} \\  }
	\only<5>{& 1: \quad \frac{\partial y}{\partial x_1} =  \frac{\partial y}{\partial z_1} \cdot {\color{red} \frac{\partial z_1}{\partial x_1}}  + \frac{\partial y}{\partial z_2} \cdot {\color{green} \frac{\partial z_2}{\partial x_1}} \\  }
	\only<6>{& 1: \quad \frac{\partial y}{\partial x_1} =  \frac{\partial y}{\partial z_1} \cdot {\color{red} \frac{\partial z_1}{\partial x_1}}  + \frac{\partial y}{\partial z_2} \cdot {\color{green} \frac{\partial z_2}{\partial x_1}} \qquad \frac{\partial y}{\partial x_2} =  \frac{\partial y}{\partial z_1} \cdot {\color{red} \frac{\partial z_1}{\partial x_2}}  + \frac{\partial y}{\partial z_2} \cdot {\color{green} \frac{\partial z_2}{\partial x_2}}  \\  }
	\only<7>{
		& 3: \quad  {\color{blue} \frac{\partial y}{\partial h_2}   \qquad  \boxed{ \frac{\partial y}{\partial h_1}} } \\
		& 2: \quad \boxed{ \frac{\partial y}{\partial z_1} } = {\color{blue}   \boxed{  \frac{\partial y}{\partial h_1}}}  \cdot {\color{amethyst} \frac{\partial h_1}{\partial z_1}} + \frac{\partial y}{\partial h_2} \cdot {\color{junglegreen} \frac{\partial h_2}{\partial z_2}}   \qquad  \frac{\partial y}{\partial z_2} =  {\color{blue} \boxed{ \frac{\partial y}{\partial h_1}} } \cdot {\color{amethyst} \frac{\partial h_1}{\partial z_2}}  + {\color{blue} \frac{\partial y}{\partial h_2}} \cdot {\color{junglegreen} \frac{\partial h_2}{\partial z_2}} \\
		& 1: \quad \frac{\partial y}{\partial x_1} =  \boxed{ \frac{\partial y}{\partial z_1} } \cdot {\color{red} \frac{\partial z_1}{\partial x_1}}  + \frac{\partial y}{\partial z_2} \cdot {\color{green} \frac{\partial z_2}{\partial x_1}} \qquad \frac{\partial y}{\partial x_2} = \boxed{ \frac{\partial y}{\partial z_1}} \cdot {\color{red} \frac{\partial z_1}{\partial x_2}}  + \frac{\partial y}{\partial z_2} \cdot {\color{green} \frac{\partial z_2}{\partial x_2}}  \\  
	}
\end{aligned} 
\end{equation*}

\begin{center}
	\begin{tikzpicture}
	\tikzstyle{place}=[circle, draw=black, minimum size = 8mm]
	
	% Input
	\foreach \x in {1,...,2}
	\draw node at (0, -\x*1.5) [place] (first_\x) {$x_\x$};
	
	% Hidden 1
	\foreach \x in {1,...,2}
	\node at (3, -\x*1.5) [place] (second_\x){$z_\x$};		
	
	% Hidden 2
	\foreach \x in {1,...,2}
	\node at (6, -\x*1.5) [place] (third_\x){$h_\x$};	
	
	% Output
	\node at (9, -2.25) [place] (fourth){$y$};
	
	\draw [->, dashed, red]  (second_1) to (first_1);
	\draw [->, dashed, green]  (second_2) to (first_1);
	\draw [->, dashed, red]  (second_1) to (first_2);
	\draw [->, dashed, green]  (second_2) to (first_2);
	
	\draw [->, dashed, amethyst]  (third_1) to (second_1);
	\draw [->, dashed, junglegreen]  (third_2) to (second_1);
	\draw [->, dashed, amethyst] (third_1) to (second_2);
	\draw [->, dashed, junglegreen]  (third_2) to (second_2);
	
	\draw [->, dashed,  blue]  (fourth) to (third_1);
	\draw [->, dashed, blue]   (fourth) to (third_2);
	
	\only<1-6>{ 
		\draw [->, dashed,  ultra thick, blue]  (fourth) to (third_1);
		\draw [->, dashed, ultra thick, blue]   (fourth) to (third_2);
	}

	\only<2-6>{ 
		\draw [->, dashed, ultra thick, amethyst]  (third_1) to (second_1);
		\draw [->, dashed, ultra thick, junglegreen]  (third_2) to (second_1);
	}

	\only<3-6>{ 
		\draw [->, dashed, ultra thick, red]  (second_1) to (first_1);
	}
	\end{tikzpicture}
\end{center}
\end{frame}


\begin{frame}{Обратное распространение ошибки}
\begin{wideitemize} 
	\item Это называется reverse-mode дифференцирование, в теории нейросетей это называют \alert{back-propagation (обратное распространение ошибки)}
	
	\item Работает быстро, потому что переиспользует вычисленные ранее значения
	
	\item На самом деле, по каждому ребру пройдемся всего раз, то есть сложность линейна	по количеству ребер (т.е. параметров)
\end{wideitemize} 
\end{frame}


\begin{frame}{Back-propagation на одном нейроне}
\begin{center}
	\begin{tikzpicture}
	\tikzstyle{place}=[circle, draw=black, minimum size = 8mm]
	\tikzstyle{place2}=[circle, draw=red, minimum size = 8mm]
	
	% Input
	\foreach \x in {1,...,2}
	\draw node at (0, -\x*1.5) [place] (first_\x) {$x_\x$};
	
	% weights
	\draw node at (1, -0.5) [place] (param_1) {$w_1$};
	\draw node at (1, -4) [place] (param_2) {$w_2$};
	
	% product
	\foreach \x in {1,...,2}
	\node at (3, -\x*1.5) [place] (second_\x){$\cdot$};		
	
	\node[thick] at (7, -1) {$z = \sigma(s) = \sigma(w_1 x_1 + w_2 x_2)$};
	
	% sum
	\node  at (5, -2.25) [place] (sum){$s$};
	\node at (7, -2.25) [place] (activ){$\sigma$};
	\node at (9, -2.25) [place] (out){$z$};
	
	\draw [->]  (first_1) to (second_1);
	\draw [->]  (first_2) to (second_2);	
	\draw [->]  (param_1) to (second_1);
	\draw [->]  (param_2) to (second_2);
	
	\draw [->] (second_1) to (sum);
	\draw [->] (second_2) to (sum);
	
	\draw [->] (sum) to (activ);
	\draw [->] (activ) to (out);
\end{tikzpicture}
\end{center}
	
\vfill 
	
Данные текут сквозь нейрон:

\[  \boxed{X} \Rightarrow \boxed{s = X \cdot W}  \Rightarrow  \boxed{z =\sigma(s)}   \Rightarrow  \boxed{L(z, y) = (y - z)^2 }\]
\end{frame} 


\begin{frame}{Back-propagation на одном нейроне}

Forward pass:

\only<1>{ \[  \boxed{X} \Rightarrow \boxed{s = X \cdot W}  \Rightarrow  \boxed{z =\sigma(s)}   \Rightarrow  \boxed{L(z, y) = (y - z)^2 }\] }

\only<2>{ \[  \boxed{X} \Rightarrow { \color{amethyst} \boxed{s = X \cdot W} }  \Rightarrow  \boxed{z =\sigma(s)}   \Rightarrow  \boxed{L(z, y) = (y - z)^2 }\] }

Backward pass: 

\begin{center}
	\begin{tikzpicture}
	\tikzstyle{place}=[circle, draw=black, minimum size = 8mm]
	
	% Input
	\foreach \x in {1,...,2}
	\draw node at (0, -\x*1.5) [place] (first_\x) {$x_\x$};
	
	% weights
	\draw node at (1, -0.5) [place] (param_1) {$w_1$};
	\draw node at (1, -4) [place] (param_2) {$w_2$};
	
	% product
	\foreach \x in {1,...,2}
	\node at (3, -\x*1.5) [place] (second_\x){$\cdot$};		
	
	% sum
	\node at (5, -2.25) [place] (sum){$s$};
	\node at (7, -2.25) [place] (activ){$f$};
	\node at (9, -2.25) [place] (out){$z$};
	\node at (11, -2.75) [place] (loss){$L$};
	\node at (9, -3.25) [place] (target){$y$};
	
	\only<2>{
		\node[amethyst,  thick] at (7, 0.4) {\text{Нам нужно вычислить} };
		\node[amethyst,  thick] at (7, 0) {\text{сигмоиду в точке s} };
		
		\node[thick] at (8, -1) {$\frac{\partial \sigma}{\partial s} = \sigma(s) \cdot (1 - \sigma(s))$};
	}
	
	\draw [->, dashed, red]  (second_1) to (first_1) node[below right=2.mm] {$w_1$} ;
	\draw [->, dashed, red]  (second_2) to (first_2) node[above right=2.mm] {$w_2$} ;
	\draw [->, dashed, red]  (second_1) to (param_1) node[right=1.cm] {$x_1$} ;
	\draw [->, dashed, red]  (second_2) to (param_2) node[right=1.cm] {$x_2$} ;
	
	\draw [->, dashed, red] (sum) to (second_1)  node[right=8.mm] {$1$} ;
	\draw [->, dashed, red] (sum) to (second_2)  node[right=8.mm] {$1$} ;
	
	\only<1>{ \draw [->, dashed, red] (activ) to (sum)  node[right=6.mm] {$\frac{\partial \sigma}{\partial s}$} ; }
	
	\only<2>{ \draw [->, dashed, ultra thick, amethyst] (activ) to (sum)  node[right=6.mm] {$\frac{\partial \sigma}{\partial s}$} ; }
	
	\draw [->, dashed, red] (out) to (activ)  node[right=7.mm] {$1$} ;
	\draw [->, dashed] (loss) to (out) node[right=7.mm] {$\frac{\partial L}{\partial z}$} ;
	\draw [->, dashed] (loss) to (target) ;
	\end{tikzpicture}
\end{center}
\end{frame} 


\begin{frame}{Сигмоида: прямой проход (forward pass)}
\begin{center}
	\begin{tikzpicture}
	\tikzstyle{place}=[circle, draw=black, minimum size = 8mm]
	\tikzstyle{place2}=[circle, draw=red, minimum size = 8mm]
	
	% Input
	\foreach \x in {1,...,2}
	\draw node at (0, -\x*1.5) [place] (first_\x) {$x_\x$};
	
	% weights
	\draw node at (1, -0.5) [place] (param_1) {$w_1$};
	\draw node at (1, -4) [place] (param_2) {$w_2$};
	
	% product
	\foreach \x in {1,...,2}
	\node at (3, -\x*1.5) [place] (second_\x){$\cdot$};		
	
	\node[thick] at (7, -1) {$z = \sigma(s) = \sigma(w_1 x_1 + w_2 x_2)$};
	
	% sum
	\node  at (5, -2.25) [place2] (sum){\color{red} $s$};
	\node at (7, -2.25) [place2] (activ){\color{red} $\sigma$};
	\node at (9, -2.25) [place2] (out){\color{red} $z$};
	
	\draw [->]  (first_1) to (second_1);
	\draw [->]  (first_2) to (second_2);	
	\draw [->]  (param_1) to (second_1);
	\draw [->]  (param_2) to (second_2);
	
	\draw [->] (second_1) to (sum);
	\draw [->] (second_2) to (sum);
	
	\draw [->, red] (sum) to (activ);
	\draw [->, red] (activ) to (out);
	\end{tikzpicture}
\end{center}
	
{\color{green} def}  {\color{blue} forward\_pass}(s):  \\
\mbox{ } \hspace{5mm} {\color{green} return } 1/(1 + np.exp(-s))
\end{frame} 



\begin{frame}{Сигмоида: обратный проход (backward pass)}
\begin{center}
	\begin{tikzpicture}
	\tikzstyle{place}=[circle, draw=black, minimum size = 8mm]
	\tikzstyle{place2}=[circle, draw=red, minimum size = 8mm]
	\tikzstyle{place3}=[circle, draw=amethyst, minimum size = 8mm]
	
	% Input
	\foreach \x in {1,...,2}
	\draw node at (0, -\x*1.5) [place] (first_\x) {$x_\x$};
	
	% weights
	\draw node at (1, -0.5) [place] (param_1) {$w_1$};
	\draw node at (1, -4) [place] (param_2) {$w_2$};
	
	% product
	\foreach \x in {1,...,2}
	\node at (3, -\x*1.5) [place] (second_\x){$\cdot$};		
	
	% sum
	\node at (5, -2.25) [place3] (sum){\color{amethyst} $s$};
	\node at (7, -2.25) [place2] (activ){\color{red} $\sigma$};
	\node at (9, -2.25) [place2] (out){\color{red} $z$};
	\node at (11, -2.75) [place] (loss){$L$};
	\node at (9, -3.25) [place] (target){$y$};
	
	\node[amethyst,  thick] at (7, 0.4) {\text{Нам нужно вычислить} };
	\node[amethyst,  thick] at (7, 0) {\text{сигмоиду в точке s} };
	
	\node[thick] at (8, -1) {\color{red} $\frac{\partial \sigma}{\partial s} = \sigma(s) \cdot (1 - \sigma(s))$};
	
	\draw [->, dashed]  (second_1) to (first_1) node[below right=2.mm] {$w_1$} ;
	\draw [->, dashed]  (second_2) to (first_2) node[above right=2.mm] {$w_2$} ;
	\draw [->, dashed]  (second_1) to (param_1) node[right=1.cm] {$x_1$} ;
	\draw [->, dashed]  (second_2) to (param_2) node[right=1.cm] {$x_2$} ;
	
	\draw [->, dashed] (sum) to (second_1)  node[right=8.mm] {$1$} ;
	\draw [->, dashed] (sum) to (second_2)  node[right=8.mm] {$1$} ;
	
	 \draw [->, dashed, red] (activ) to (sum)  node[right=6.mm] {$\frac{\partial \sigma}{\partial s}$} ; 
	
	\draw [->, dashed, red] (out) to (activ)  node[right=7.mm] {$1$} ;
	\draw [->, dashed, junglegreen] (loss) to (out) node[right=7.mm] {\color{junglegreen} $\frac{\partial L}{\partial z}$} ;
	\draw [->, dashed] (loss) to (target) ;
	\end{tikzpicture}
\end{center}

\begin{columns}
	\begin{column}{.69\textwidth}
		{\color{green} def}  {\color{blue} backward\_pass}({\color{amethyst} s}, {\color{junglegreen} incoming\_gradient}):  \\
		\mbox{ } \hspace{5mm} { sigm =  1/(1 + np.exp({\color{amethyst}-s}))}) \\
		\mbox{ } \hspace{5mm} {\color{green} return }  {\color{red} sigm * (1 - sigm) } *  {\color{junglegreen} incoming\_gradient}
	\end{column}
	\begin{column}{.29\textwidth}
		\[  \frac{ \partial L}{\partial s}  = {\color{red}  \frac{ \partial \sigma}{\partial s} } \cdot {\color{junglegreen} \frac{ \partial L}{\partial \sigma} }   \]
	\end{column}
\end{columns}
\end{frame} 


\begin{frame}{Полносвязный слой: прямой проход (forward pass)}
\begin{wideitemize}
\item Два нейрона с тремя входами:

\begin{equation*}
\begin{aligned}
	& z_1 = x_1 w_{11} + x_2 w_{21} + x_3 w_{31}  \\  
	& z_2 = x_1 w_{12} + x_2 w_{22} + x_3 w_{32}  \\  
\end{aligned}
\end{equation*}

\item Матричная запись: 

\begin{equation*}
	\begin{pmatrix} z_1 & z_2 \end{pmatrix}  = \begin{pmatrix} x_1 & x_2 & x_3  \end{pmatrix}  \cdot  \begin{pmatrix} w_{11} & w_{12} \\ w_{21} & w_{22} \\ w_{31}  & w_{32} \\  \end{pmatrix}
\end{equation*}

\[ 
	z  = x W 
\]

\end{wideitemize}
\end{frame}


\begin{frame}{Полносвязный слой: обратный проход (backward pass)}
\begin{wideitemize}

	\item Матричная запись: 
	
	\begin{equation*}
	\begin{pmatrix} z_1 & z_2 \end{pmatrix}  = \begin{pmatrix} x_1 & x_2 & x_3  \end{pmatrix}  \cdot  \begin{pmatrix} w_{11} & w_{12} \\ w_{21} & w_{22} \\ w_{31}  & w_{32} \\  \end{pmatrix}
	\end{equation*}
	
	\[ 
	Z  = X W 
	\]

\only<1>{	
	\item Для обратного прохода нам нужна $\frac{\partial L}{\partial W}$: 
	
	\[
	W_t = W_{t-1} - \eta_t \cdot \left. \frac{\partial L}{\partial W} \right |_{W_{t-1}}
	\]
	}

%\only<2>{	
%	\item Нужная нам производная - матрица: 
%	
%	\begin{equation*}
%		\frac{\partial L}{\partial W} =  \begin{pmatrix}   \frac{\partial L}{\partial w_{11}} & \frac{\partial L}{\partial w_{12}} \\ \frac{\partial L}{\partial w_{21}} & \frac{\partial L}{\partial w_{22}} \\ \frac{\partial L}{\partial w_{31}} & \frac{\partial L}{\partial w_{32}} \\  \end{pmatrix}
%	\end{equation*}
%}
\end{wideitemize}
\end{frame}


%\begin{frame}{Полносвязный слой: обратный проход (backward pass)}
%\begin{wideitemize}
%	\item Применим цепное правило: 
%	
%	\[
%	\frac{\partial L}{\partial w_{ij}} =  \sum_k \frac{\partial L}{\partial z_k} \cdot \frac{\partial z_k}{\partial w_{ij}} =   \frac{\partial L}{\partial z_j} \cdot x_i
%	\]
%	
%	\[
%	z_j = x_1 w_{1j} + x_2 w_{2j} + x_3 w_{3j} 
%	\]
%	
%	\only<2>{
%	\item перепишем в матричном виде: 
%	
%	\[
%	\frac{\partial L}{\partial W} = \begin{pmatrix}  x_1 \\ x_2 \\ x_3 \end{pmatrix}  \cdot  \begin{pmatrix}  \frac{\partial L}{\partial z_1}  &  \frac{\partial L}{\partial z_2} \end{pmatrix} = x^T \cdot \frac{\partial L}{\partial z}
%	\]
%	}
%\end{wideitemize}
%\end{frame}


\begin{frame}[fragile]{Полносвязный слой в numpy}

\alert{Прямой проход:} 

\mbox{  }

\begin{columns}
\begin{column}{.49\textwidth}
{\color{green} def}  {\color{blue} forward\_pass}(X, W):  \\
\mbox{ } \hspace{5mm} {\color{green} return } X.dot(W))
\end{column}
\begin{column}{.49\textwidth}
\[ Z = XW \]
\end{column}
\end{columns}

\vfill 

\alert{Обратный проход:}

\begin{columns}
\begin{column}{.49\textwidth}
{\color{green} def}  {\color{blue} forward\_pass}(X, W, in\_grad):  \\
\mbox{ } \hspace{5mm} dX = in\_grad.dot(W.T) \\
\mbox{ } \hspace{5mm} dW = X.T.dot(in\_grad) \\
\mbox{ } \hspace{5mm} {\color{green} return } dX, dW
\end{column}
\begin{column}{.49\textwidth}

\[  \frac{\partial L}{\partial X}  =   \frac{\partial L}{\partial Z} \cdot W^T \]
\[ \frac{\partial L}{\partial W}  =  X^T \cdot \frac{\partial L}{\partial Z} \]
\end{column}
\end{columns}
\begin{center}
	\alert{Эти формулы мы получим на семинаре}
\end{center}
\end{frame}


\begin{frame}{Back-propagation}
\begin{center}
	\includegraphics[width=.8\linewidth]{backpropagation.png}
\end{center}
\end{frame}


\begin{frame}{Back-propagation}

\alert{Forward pass:}

\begin{center}
	\begin{tikzpicture}
	\tikzstyle{place}=[rectangle, draw=black, minimum size = 8mm]
	\draw node at (0, 0) (input) {$X$};
	\draw node at (2, 0) (h1) {$H_1$};
	\draw node at (4, 0) (o1) {$O_1$};
	\draw node at (6, 0) (h2) {$H_2$};
	\draw node at (8, 0) (o2) {$O_2$};
	\draw node at (10, 0) (output) {$\hat{y} $};
	\draw node at (12, 0) (mse) {$MSE$};
	
	\draw [->]  (input) -- (h1) node[pos=.49, above] {$W_1$} ;
	\draw [->]  (h1) -- (o1) node[pos=.49, above] {$f$} ;
	\draw [->]  (o1) -- (h2) node[pos=.49, above] {$W_2$} ;
	\draw [->]  (h2) -- (o2) node[pos=.49, above] {$f$} ;
	\draw [->]  (o2) -- (output) node[pos=.49, above] {$W_3$} ;
	\draw [->]  (output) to (mse);	
	\end{tikzpicture}
\end{center}
	

\vfill
	
\alert{Backward pass:}
	
\begin{center}
	\begin{tikzpicture}
		\draw node at (0, 0) (input) {$X$};
		\draw node at (2, 0) (h1) {$H_1$};
		\draw node at (4, 0) (o1) {$O_1$};
		\draw node at (6, 0) (h2) {$H_2$};
		\draw node at (8, 0) (o2) {$O_2$};
		\draw node at (10, 0) (output) {$\hat{y} $};
		\draw node at (12, 0) (mse) {$MSE$};
		
		\draw [->, dashed]   (h1)  -- (input) node[pos=.49, above] {$\frac{\partial H_1}{\partial X}$} ;
		\draw [->, dashed]   (o1) -- (h1) node[pos=.49, above] {$\frac{\partial O_1}{\partial H_1}$} ;
		\draw [->, dashed]   (h2) -- (o1) node[pos=.49, above] {$\frac{\partial H_2}{\partial O_1}$} ;
		\draw [->, dashed]   (o2) -- (h2) node[pos=.49, above] {$\frac{\partial O_2}{\partial H_2}$} ;
		\draw [->, dashed]   (output) -- (o2)node[pos=.49, above] {$\frac{\partial \hat{y}}{\partial O_2}$} ;
		\draw [->, dashed]  (mse) -- (output) node[pos=.49, above] {$\frac{\partial MSE}{\partial \hat{y} }$} ;	
		
		\draw [->, dashed]  (9, -0.05) -- (9, -1.2)  node[pos=.49, right] {$\frac{\partial \hat{y}}{\partial W_3} = O_2^T$} ;
		\draw [->, dashed]  (5, -0.05) -- (5, -1.2)  node[pos=.49, right] {$\frac{\partial H_2}{\partial W_2} = O_1^T$} ;
		\draw [->, dashed]  (1, -0.05) -- (1, -1.2)  node[pos=.49, right] {$\frac{\partial H_1}{\partial W_1} = X^T$} ;
	\end{tikzpicture}
\end{center}
\end{frame}



\begin{frame}{Back-propagation}
\alert{Forward pass:}

\begin{center}
	\begin{tikzpicture}
	\tikzstyle{place}=[rectangle, draw=black, minimum size = 8mm]
	\draw node at (0, 0) (input) {$X$};
	\draw node at (2, 0) (h1) {$H_1$};
	\draw node at (4, 0) (o1) {$O_1$};
	\draw node at (6, 0) (h2) {$H_2$};
	\draw node at (8, 0) (o2) {$O_2$};
	\draw node at (10, 0) (output) {$\hat{y} $};
	\draw node at (12, 0) (mse) {$MSE$};
	
	\draw [->]  (input) -- (h1) node[pos=.49, above] {$W_1$} ;
	\draw [->]  (h1) -- (o1) node[pos=.49, above] {$\sigma$} ;
	\draw [->]  (o1) -- (h2) node[pos=.49, above] {$W_2$} ;
	\draw [->]  (h2) -- (o2) node[pos=.49, above] {$\sigma$} ;
	\draw [->]  (o2) -- (output) node[pos=.49, above] {$W_3$} ;
	\draw [->]  (output) to (mse);	
	\end{tikzpicture}
\end{center}

\vfill

\alert{Backward pass:}

\begin{center}
	\begin{tikzpicture}
	\draw node at (0, 0) (input) {$X$};
	\draw node at (2, 0) (h1) {$H_1$};
	\draw node at (4, 0) (o1) {$O_1$};
	\draw node at (6, 0) (h2) {$H_2$};
	\draw node at (8, 0) (o2) {$O_2$};
	\draw node at (10, 0) (output) {$\hat{y}$};
	\draw node at (12, 0) (mse) {$MSE$};
	
	\draw [->, dashed]   (h1)  -- (input) node[pos=.49, above] {\scriptsize $W_1^T$} ;
	\draw [->, dashed]   (o1) -- (h1) node[pos=.49, above] {\scriptsize $O_1 (1-O_1)$} ;
	\draw [->, dashed]   (h2) -- (o1) node[pos=.49, above] {\scriptsize $W_2^T$} ;
	\draw [->, dashed]   (o2) -- (h2) node[pos=.49, above] {\scriptsize $O_2 (1 - O_2)$} ;
	\draw [->, dashed]   (output) -- (o2)node[pos=.49, above] {\scriptsize $W_3^T$} ;
	\draw [->, dashed]  (mse) -- (output) node[pos=.49, above] {\scriptsize $-2(\hat{y} - y)$} ;	
	
	\draw [->, dashed]  (9, -0.05) -- (9, -1.2)  node[pos=.49, right] {$O_2^T$} ;
	\draw [->, dashed]  (5, -0.05) -- (5, -1.2)  node[pos=.49, right] {$O_1^T$} ;
	\draw [->, dashed]  (1, -0.05) -- (1, -1.2)  node[pos=.49, right] {$X^T$} ;
	\end{tikzpicture}
\end{center}
\end{frame}



\begin{frame}{Back-propagation}

\begin{center}
	\begin{tikzpicture}
	\draw node at (0, 0) (input) {$X$};
	\draw node at (2, 0) (h1) {$H_1$};
	\draw node at (4, 0) (o1) {$O_1$};
	
	\only<3>{ 
		\draw node at (0, 0) (input) {\color{red} $X$};
		\draw node at (2, 0) (h1) {\color{red} $H_1$};
		\draw node at (4, 0) (o1) {\color{red} $O_1$};
	}
	
	\draw node at (6, 0) (h2) {$H_2$};
	\draw node at (8, 0) (o2) {$O_2$};
	
	\only<2>{ 
		\draw node at (6, 0) (h2) {\color{red}  $H_2$};
		\draw node at (8, 0) (o2) {\color{red} $O_2$};	
	}
	
	\draw node at (10, 0) (output) {$\hat{y}$};
	\draw node at (12, 0) (mse) {$MSE$};
	
	\only<1>{ 
		\draw node at (10, 0) (output) {\color{red} $\hat{y}$};
		\draw node at (12, 0) (mse) {\color{red} $MSE$};	
	}
		
	\draw [->, dashed]   (output) -- (o2)node[pos=.49, above] {\scriptsize $W_3^T$} ;
	\draw [->, dashed]  (mse) -- (output) node[pos=.49, above] {\scriptsize $-2(\hat{y} - y)$} ;	
	\draw [->, dashed]  (9, -0.05) -- (9, -1.2)  node[pos=.49, right] {$O_2^T$} ;
		
	\only<1>{ 
		\draw [->, dashed, red]   (output) -- (o2)node[pos=.49, above] {\color{red} \scriptsize $W_3^T$} ;
		\draw [->, dashed, red]  (mse) -- (output) node[pos=.49, above] {\color{red} \scriptsize $-2(\hat{y} - y)$} ;	
		\draw [->, dashed, red]  (9, -0.05) -- (9, -1.2)  node[pos=.49, right] {\color{red} $O_2^T$} ;
	}

	\draw [->, dashed]   (h2) -- (o1) node[pos=.49, above] {\scriptsize $W_2^T$} ;
	\draw [->, dashed]   (o2) -- (h2) node[pos=.49, above] {\scriptsize $O_2 (1 - O_2)$} ;
	\draw [->, dashed]  (5, -0.05) -- (5, -1.2)  node[pos=.49, right] {$O_1^T$} ;
	
	\only<2>{ 
		\draw [->, dashed, red]   (h2) -- (o1) node[pos=.49, above] {\color{red} \scriptsize $W_2^T$} ;
		\draw [->, dashed, red]   (o2) -- (h2) node[pos=.49, above] {\color{red} \scriptsize $O_2 (1 - O_2)$} ;
		\draw [->, dashed, red]  (5, -0.05) -- (5, -1.2)  node[pos=.49, right] {\color{red} $O_1^T$} ;
	}

	\draw [->, dashed]   (h1)  -- (input) node[pos=.49, above] {\scriptsize $W_1^T$} ;
	\draw [->, dashed]   (o1) -- (h1) node[pos=.49, above] {\scriptsize $O_1 (1-O_1)$} ;
	\draw [->, dashed]  (1, -0.05) -- (1, -1.2)  node[pos=.49, right] {$X^T$} ;

	\only<3>{ 
		\draw [->, dashed, red]   (h1)  -- (input) node[pos=.49, above] {\color{red} \scriptsize $W_1^T$} ;
		\draw [->, dashed, red]   (o1) -- (h1) node[pos=.49, above] {\color{red} \scriptsize $O_1 (1-O_1)$} ;
		\draw [->, dashed, red]  (1, -0.05) -- (1, -1.2)  node[pos=.49, right] {\color{red} $X^T$} ;
	}	
	\end{tikzpicture}
\end{center}

\vfill

\begin{columns}
	\begin{column}{.28\textwidth}
		\alert{Шаг 1:}
		\begin{equation*} 
			\begin{aligned}
				&  d = -2 (\hat{y} - y) \\
				&  \frac{\partial MSE}{\partial W_3} = O_2^T \cdot d    \\
			\end{aligned}
		\end{equation*}
	\end{column}
	\begin{column}{.3\textwidth}
		\only<2-4>{
		\alert{Шаг 2:}
		\begin{equation*} 
			\begin{aligned}
				&  d = d \cdot W_3^T * O_2 * (1 - O_2)  \\
				&  \frac{\partial MSE}{\partial W_2} = O_1^T \cdot d \\
			\end{aligned}
		\end{equation*}}
	\end{column}
	\begin{column}{.35\textwidth}
		\only<3-4>{
		\alert{Шаг 3:}
		\begin{equation*} 
		\begin{aligned}
		&  d = d \cdot W_2^T * O_1 * (1 - O_1)  \\
		&  \frac{\partial MSE}{\partial W_1} = X^T \cdot d \\
		\end{aligned}
		\end{equation*}}
	\end{column}
\end{columns}

\vfill

\only<4>{  \alert{  Шаг SGD:}
{\footnotesize
\begin{columns}
	\begin{column}{.28\textwidth}
			\[  W_3^t = W_3^{t-1} - \eta \cdot \frac{\partial MSE}{\partial W_3}  \]
	\end{column}
	\begin{column}{.3\textwidth}
			\[  W_2^t = W_2^{t-1} - \eta \cdot \frac{\partial MSE}{\partial W_2}  \]
	\end{column}
	\begin{column}{.35\textwidth}
			\[  W_1^t = W_1^{t-1} - \eta \cdot \frac{\partial MSE}{\partial W_1}  \]
	\end{column}
\end{columns} 
	}
}
\end{frame}


\begin{frame}{Численная оценка производной}
	\begin{wideitemize}
		\item  \alert{Способа считать производную быстрее —  нет!} 
		
		\item Можно попробовать посчитать численную оценку
		
		\[
		\frac{f(x, w + \varepsilon) -  f(x, w)}{ \varepsilon}
		\]
		
		\item При больших $ \varepsilon$ результат будет очень неточным, при малых  $\varepsilon$ начнутся \alert{численные приколы} с точностью вычислений
		
		\item На практике лучше работает формула
		
		\[
		\frac{f(x, w + \varepsilon) -  f(x, w -  + \varepsilon)}{ 2 \cdot \varepsilon}
		\]
		
	\end{wideitemize}
\end{frame}


\begin{frame}{Что такое слой в нейронной сети?}
	\begin{wideitemize}
		\item Любой слой - это какая-то абстракция, которая умеет делать прямой шаг и обратный шаг
	
		\item Для всех слоёв, которые мы дальше будем изучать, мы всегда будем смотреть на то как выглядят эти два шага
	\end{wideitemize}
\end{frame}


\begin{frame}{А мне точно надо понимать backprop?}
	\begin{wideitemize}
		\item  \alert{Да, точно!} 
				
		\item  "Backprop – leaky abstraction!"
		
		\item  Почему сеть не обучается?
		
		\item  Почему сеть обучается слишком медленно? 
		
		\item  Какие проблемы могут возникать в обучении из-за плохой архитектуры? 
	\end{wideitemize}
\end{frame}


\end{document}

